\documentclass[conference,10pt]{IEEEtran}
%\documentclass[conference,draft,onecolumn]{IEEEtran}
% useful packages, copy and paste from diff sources

\usepackage[english]{babel}
\usepackage[T1]{fontenc}
\usepackage{cite,url,color} % Citation numbers being automatically sorted and properly "compressed/ranged".
\usepackage{graphics,amsfonts}
\usepackage{epstopdf}
\usepackage[pdftex]{graphicx}
\usepackage[cmex10]{amsmath}
% Also, note that the amsmath package sets \interdisplaylinepenalty to 10000
% thus preventing page breaks from occurring within multiline equations. Use:
\interdisplaylinepenalty=2500
% after loading amsmath to restore such page breaks as IEEEtran.cls normally does.
\usepackage[utf8]{inputenc}
% Useful for displaying quotations
%\usepackage{csquotes}
% Compact lists
%\let\labelindent\relax
\usepackage{enumitem}

%tikz figures
\usepackage{tikz}
\usetikzlibrary{automata,positioning,chains,shapes,arrows}
\usepackage{pgfplots}
\usetikzlibrary{plotmarks}
\newlength\fheight
\newlength\fwidth
\pgfplotsset{compat=newest}
\pgfplotsset{plot coordinates/math parser=false}

\usepackage{array}
% http://www.ctan.org/tex-archive/macros/latex/required/tools/
%\usepackage{mdwmath}
%\usepackage{mdwtab}
%mdwtab.sty	-- A complete ground-up rewrite of LaTeX's `tabular' and  `array' environments.  Has lots of advantages over
%		   the standard version, and over the version in `array.sty'.
% *** SUBFIGURE PACKAGES ***
%\usepackage[tight,footnotesize]{subfigure}
\usepackage{subfig}

\usepackage[top=1.5cm, bottom=2cm, right=1.6cm,left=1.6cm]{geometry}
\usepackage{indentfirst}

\usepackage{times}
% make sections titles smaller to save space
%\usepackage{sectsty}
%\sectionfont{\large}
% enable the use of 'compactitem', a smaller 'itemize'
%\usepackage{paralist}

% MP
% to split equations using dmath env
\usepackage{breqn}
% nice rules in tables
\usepackage{booktabs}

%\setlength\parindent{0pt}
\linespread{1}

% MC
\newcommand{\MC}[1]{\textit{\color{red}MC says: #1}}
\newcommand{\AZ}[1]{\textit{\color{blue}AZ says: #1}}
\newcommand{\MP}[1]{\textit{\color{green}MP says: #1}}

\usepackage{placeins}


%%%%%%%%%%%%%%%%%%%%%%%%%%%%%%%%%%%%%%%%%%
\begin{document}
%%%%%%%%%%%%%%%%%%%%%%%%%%%%%%%%%%%%%%%%%%
\title{Quality of Service with Software Defined Network.}

\author{\IEEEauthorblockN{Andrea Pittaro, Matteo Maso }
\IEEEauthorblockA{Department of Information Engineering, University of Padova -- Via Gradenigo, 6/b, 35131 Padova, Italy\\Email: {\tt\{pittaroa,masomatt\}@dei.unipd.it}
}}

\maketitle

\begin{abstract}
Quality of Service(QoS) in existing network architectures is an ongoing problem.
A lot of researcher and industries try to solve this problem with many solution but a lot of them are too 
expensive other failed or are not implemented.
Software Defined Networking(SDN) paradigm has emerged in response to limitations of traditional networking architectures.
The main advantages of SDN are centralized global network view. This particular features have got attention of researchers to improve QoS.
In questo report mostreremo quali sono stati i principali risutati raggiunti e i problemi incontrati nella realizzazione di 
un applicativo SDN in grado di supportare un servizio di tipo QoS.
Lo standard OpenFlow e il controller python POX largamente diffuso sono stati gli strumenti che hanno reso possibile la
realizzazione di quest'esperienza pur mostrando i limiti dovuti all'utilizzo di una tecnologia nuova e in fase di ricerca.
%translate
\end{abstract}

\section{Introduction}\label{sec:intro}
Software-defined networking (SDN) is way to manage networks that separates the control plane from the forwarding plane. 
SDN offers a centralized view of the network, giving an SDN Controller the ability to act as the “brains” of the network. 
The SDN Controller relays information to switches and routers via southbound APIs, and to the applications with northbound APIs. 
One of the most well-known protocols used by SDN Controllers is OpenFlow, however, it isn’t the only SDN standard, but it is widely accepted.
Centralized, programmable SDN environments can easily adjust to the rapidly changing needs of businesses. SDN can lower costs and limit
wasteful provisioning, as well as provide flexibility and innovation for networks.
\#TODO da completare 
  \subsection{OpenFlow}
  During 2006 thanks to M. Casado was developed Ethane a new network architecture on this conncept during 2011 was bord the Open Network Foundatin.
  This is a no profit association that would innovate SDN and standardize OpenFlow protocol whit relatives tecnology \cite{ONF}.
  The first version of OpenFlow is 1.0.0 release on 2009 but the last standardized version is v1.5.1 approved on march 2015 \cite{ONF_report}.
  \#TODO da sistemare
  \subsection{The state of the art}
  \#TODO stato dell arte e paper di riferimento  matteo

\section{Needs and Gools}\label{sec:obb} %obbiettivi ed esigenze controllare 
The main gool of this experience is to obtain statistics on realtime traffic in order to reroute traffic based on QoS or other reasons. 
%correggere
  \subsection{Adaptability of the network}
  Made Network Adaptable means poter dirottare il traffico in real time to ensura a lot of demands. For example we have to avoid link congestion, 
  but we can prefer reroute some hosting traffic to peggiorare it's performance because of it's evil traffic.
  To reach this gools we need to have a global knowledge on the network and we have to modify the routing table on every routerboard with a 
  global scope.
  The performance of a single connection and entire network can ensure various type of metrics.
  In order to assicurare queste ci serve una struttura dati che si aggiorna costantemente nella quale immagazzinare le statistiche dei link 
  in modo all'occorrenza di trovare il percordo migliore per ogni singolo flusso dati.
  \#TODO MATTEO
  \subsection{Exterior comunication}
  \#TODO pitt
  
\section{Instrument}\label{sec:instrument}
\#TODO matteo
  \subsection{Controller}
  The controller that we use is POX, it's repository is at this link \url{https://github.com/noxrepo/pox}. 
  POX is an opensource python controller, it is an evolution of NOX a controller developed on C language.
  The documentation about the progect is present at this link \url{https://openflow.stanford.edu/display/ONL/POX+Wiki}.
  The only problem to use this controller is that it support only the first version of OpenFlow1.0.
  
  The controller run inside a VirtualMachine on our personal computer, offered by SDN hub \cite{VM}. This VM use UBUNTU like OS and version 2.7.6 of
  Python that is sufficient for POX that's required version 2.7.
  % \#TODO matteo da correggere e completare
  \subsection{Router Board}
  The router board that we used is MikroTik miniROUTER RouterBOARD 450G. This Router Board virtualize an OpenFlow switch that support 
  OpenFlow v1.0. The main caraptheristics is that it has only 5 ethernet port this is a limitation because consider that one of this port it is 
  used to connect conntroller and with other port we have made host and other routerboard connection.
  
  %\#TODO matteo ( parlare del firmware utilizzato nella routerboard) correggere e completare
  
\section{Results}\label{sec:results}
\#TODO %qui non so se serve un'introduzione o meno
  \subsection{Graph and Network structures}
  In order to achieve the realtime adaptability and statistics storage we have develop ad-hoc structure.
  We use a specific python library called NetworkX \cite{networkx} to create a graph in witch we can storage ad-hoc node that represent 
  resplectively switch or host element.
  This topological structure is updated realtime thanks to connection event and link discover offered by various library. 
  ( mettere i vari eventi che sfrutto )
  The link besides have many parameters in witch we can save network performance. To update every link parameter we have to creat ad hoc function 
  because there isn't some function constructed yet. ( sistemare frase) .
  This type of structure allow us to run Dijkstra algorithm that return the minimum path ( in base a vari parametri ) on a nlog n time ( verificare).
  (aggiungere qualcosa e correggere)
  \#TODO matteo
  
  \subsection{Traffics analisis}
  For every networks performance we have tryed to develop a specific solution. The OpenFlow main goal is to centralize controll and software
  complexity so the routerboard can made only some easy function in orther to contain the hardware complexity and mantainly a good speed.
  Il che ha portato a dover utilizzare stratagemmi.
  \#TODO %riformulare
    \subsubsection{Delay}
    Per molte applicazioni odierne il delay e' un parametro importantissimo. 
    La rete attuale puo' conoscere il delay di un link semplicemente inviando un pachetto ICMP di tipo TimeStamp ( verificare ) 
    supportato dal protocollo openflow, questo permette di ottenere una precisione nell'ordine dei microsecondi ( verificare ).
    
    
    \#TODO MATTEO
    \subsubsection{Trougthput}
    \#TODO andrea
    \subsubsection{Link Capacity}
    \#TODO andrea
    \subsubsection{altre}
    \#TODO andrea
    
  \subsection{Route changing} %cambio rotte
  \#TODO andrea
  
  \subsection{ARP solution} %TODO questione dell'arp inutile
  During our analysis and in our tests, we prove that in a SDN mesh network, the ARP packets are useless.
  This type of packets is useful for a multipoint to multipoint connection where the connection is set between a
  HUB and a set of computers or between a bridge and a computer. The first case is no more in use in common networks as
  hiher the collision range. The bridges limits their collision range to the port. With openflow each switching unit can
  act as a bridge or as a router. In both cases analyses the header of the L2 packet L3 packet and some of the L4 header.
  If it has the rule to forward the packet, it will act as declared in the most specific matching rule. Otherwise it will
  send to the controller a PacketIn and the controller will act as programmed. If an host want to communicate to another host whithin
  the network, it sends the packet to the line and then the switch will check in its tables and execute the actions of the match found.
  For an IP network, it will check the path to reach a destination and then send to the defined output port. The mac address in this case
  became useless, as the connection is point to point. So, knowing the destination layer2 address is useless as the switch can act
  regardless of it. The only thing the switch need to know is the layer2, layer 3 addresses and, at most, the layer 4 address.
  This can be seen by a reader as a violation of the layers in the ISO/OSI stack, but to exploit the potentials of the SDN approach,
  this is necessary. TODO explain why TODO!!!!**
  \#TODO andrea
  
  \subsection{Default Gw problem} 
  The computers don't need anymore to set a gateway to reach another network as the controller and the switches will redirect the traffic
  properly, according to the rules set by the first. The controller must only know which switch ports are connected to the
  internet and then add them to the path. in this way, the controller can set a path to reach it and provide connectivity with
  a traffic shaping policy and/or differentiate the traffic from and to different networks. 
  \#TODO andrea
  
  \subsection{Ip switching in pox}
  \#TODO andrea
  
\section{Conclusions}\label{sec:conclusion}
Conclusions are a superbrief summary of what has been done and highlighting of the "take home message"

\#TODO
  

\begin{thebibliography}{99}
	\bibitem{pox} POX \url{https://openflow.stanford.edu/display/ONL/POX+Wiki}
	\bibitem{openflow} OpenFlow \url{http://flowgrammable.org/sdn/openflow/}
	\bibitem{pox_repo} POX's repository \url{https://github.com/noxrepo/pox}
	\bibitem{VM} Virtual Machine \url{http://sdnhub.org/tutorials/pox/}
	\bibitem{router_board} Router Board \url{https://www.senetic.it/product/RB450G}
	\bibitem{netoworkx} NetworkX python's library \url{https://networkx.github.io/}
	\bibitem{ONF} Open Networking Foundation \url{https://www.opennetworking.org/sdn-resources/technical-library}
	\bibitem{ONF_report} v1.5.1 OpenFlow report \url{https://www.opennetworking.org/sdn-resources/technical-library}
	
\end{thebibliography}

\end{document}
